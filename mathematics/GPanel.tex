\documentclass{tufte-handout}

\usepackage{amsmath}
\usepackage{amssymb}
\usepackage{braket}
\usepackage{enumitem}
  \newenvironment{inlinenum}
    {\begin{enumerate}
      [ itemsep    = -1mm
      , topsep     = -1mm
      , leftmargin = .5in
      ]}
    {\vskip 1mm\end{enumerate}}

\begin{document}
  \title{Generalized Fixed Effects Model for Panel Data}
  \author{Minsheng Liu}
  \maketitle

  \newcommand \Mat[1]{\mathcal{M}_{#1}(\mathbb{R})}

  Consider the following model:
  \[ y_{t, i} = \vec{x}_{t, i} \cdot \vec{\beta} 
              + \vec{\phi}_t \cdot \vec{\alpha}_i
              + \vec{\psi}_i \cdot \vec{\gamma}_t
              + \epsilon_{i, t} \, , \]
  where $i$ ranges from $1$ to $n$, $t$ ranges from $1$ to $t$,
  and $\vec{\beta} \in \mathbb{R}^m$.

  Equivalently, the model can be expressed as
  \[ \mathbf{Y} = \sum_{i = 1}^m \beta{i} \cdot \mathbf{X}_i
                + \mathbf{\Phi}\mathbf{A}
                + (\mathbf{\Psi}\mathbf{\Gamma})^{\top}
                + \mathbf{E} \, , \]
  where
  \begin{inlinenum}
    \item $\mathbf{Y}, \mathbf{X}_i, \mathbf{E} \in \Mat{i, t}$;
    \item $\mathbf{\Phi} \in \Mat{t, p}$ is given,
          $\mathbf{A} \in \Mat{p, i}$ is unknown.
          They are called as time-observable individual-specific effects (tois);
    \item $\mathbf{\Psi} \in \Mat{i, q}$ is given,
          $\mathbf{\Gamma} \in \Mat{q, t}$ is unknown.
          They are called as individual-observable time-specific effects (iots).
  \end{inlinenum}

  Note that time andindividual fixed effects can be expressed by adding a column
  of $1$s in $\Phi$ and $\Psi$, respectively.

  \newcommand \Ptois{\mathbf{P}_{\mathsf{tois}}}
  \newcommand \Piots{\mathbf{P}_{\mathsf{iots}}}
  We will apply the method of alternate projections.
  Let
  \begin{align*}
    \Ptois &= \Phi(\Phi^{\top} \Phi)^{-1}\Phi^{\top} \, , \\
    \Piots &= \Psi(\Psi^{\top} \Psi)^{-1}\Psi^{\top} \, ,
  \end{align*}
  be the projection matrices for $\Phi$ and $\Psi$ respectively.
  Define $\mathsf{tois}: \set{ 1, \hdots, n} \to (\Mat{t, i} \to \Mat{t, i})$
     and $\mathsf{iots}: \set{ 1, \hdots, t} \to (\Mat{t, i} \to \Mat{t, i})$
  as follows\footnote{%
    As in R, we use $\mathbf{M}_{,i}$ to denote the $i$-th column and 
    $\mathbf{M}_{t,}$ to denote the $t$-th row.
    The notation $\mathbf{M}[X \mapsto Y]$ means the matrix $M$, with
    its row or column as specified by $X$ replaced by $Y$.

    Both $\mathsf{tois}$ and $\mathsf{iots}$ are so-called \emph{higher order
    functions}. Namely, it produces a function $\Mat{t, i} \to \Mat{t, i}$
    with a given $i$ or $t$.
  }:
  \begin{align*}
    \mathsf{tois}(i)(\mathbf{M}) &=
      \mathbf{M}[\mathbf{M}_{,i} \mapsto
        \mathbf{M}_{,i} - \Ptois \mathbf{M}_{,i}] \, , \\
    \mathsf{iots}(t)(\mathbf{M}) &=
      \mathbf{M}[\mathbf{M}_{t,} \mapsto
        \mathbf{M}_{t,} - \mathbf{M}_{t,} \Piots^{\top}] \, .
  \end{align*}
  Essentially,
  $\mathsf{tois}(i, \mathbf{M})$ ``demeans'' the $i$-th column
  of $\mathbf{M}$---namely the individual $i$;
  $\mathsf{iots}(t, \mathbf{M})$ ``demeans'' the $t$-th row
  of $\mathbf{M}$---namely the time $t$.

  For each iteration,
  let $\mathbf{M}$ range from $\mathbf{Y}, \mathbf{X}_1, \hdots, \mathbf{Y}$.
  we have
  \[
    \hat{\mathbf{M}} = (
      (\mathsf{iots}(t) \circ \dots \circ \mathsf{iots}(1)) \circ
      (\mathsf{tois}(n) \circ \dots \circ \mathsf{tois}(1)))(\mathbf{M}) \, .
  \]

  The algorithm stops after all $\mathbf{M}$ stablizes.

\end{document}